\documentclass[12pt,conference,onecolumn]{IEEEtran}

\title{Micromouse}
\author{%
\IEEEauthorblockN{Henry Villase\~{n}or}\IEEEauthorblockA{Science \& Engineering\\Manalapan High School\\Englishtown, NJ\\425hvillasenor@frhsd.com} \and 
\IEEEauthorblockN{Aadarsh Kumar}\IEEEauthorblockA{Science \& Engineering\\Manalapan High School\\Englishtown, NJ\\425akumar@frhsd.com}}
\date{January 28, 2025}

\newcommand{\keywords}{micromouse, IEEE, mobile robot, robotics, navigation, optimal search, shortest path, STEM education}

\usepackage{hyperref}
\makeatletter
\AtBeginDocument{
\hypersetup{%
pdftitle={\@title},
pdfauthor={Henry Villasenor and Aadarsh Kumar},
pdfkeywords={\keywords}}}
\makeatother

\begin{document}
\maketitle 

\begin{abstract}
We have designed an autonomous maze-solving robot to complete a challenge known as ``Micromouse.'' Micromouse is a competition where teams design small, fully autonomous robots that can navigate and find the optimal path through a maze. Robots may not damage the course or scale it in any way. Five consecutive runs are allowed; mice use one run to traverse, memorize, and process the fastest path through the maze. Then, subsequent runs take the shortest path for maximum speed in solving. The mouse uses IR sensors to detect walls, a flood-fill algorithm to find the optimal path, and a Raspberry Pi for all processing.
\end{abstract}

\begin{IEEEkeywords}
\keywords
\end{IEEEkeywords}

\end{document}