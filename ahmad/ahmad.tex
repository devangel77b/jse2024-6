Evaluation of DW-MRI neuroimaging data and its application in analyzing the motor outcomes of patients after ischemic stroke at Rutgers NeuREI Lab


The Neurophysiology and Rehabilitation Innovations Lab (NeuREI Lab) at Rutgers School of Health Professions, in collaboration with New Jersey Medical School, aims to enhance the lives of individuals with motor and cognitive disabilities through innovative research in neurophysiology. Led by Dr. Soha Saleh, the lab focuses on using diffusion-weighted Magnetic Resonance Imaging (DW-MRI) to understand how brain imaging can predict motor recovery in patients who have suffered ischemic strokes. DW-MRI is a non-invasive imaging technique that tracks the movement of water molecules in brain tissues, helping researchers analyze critical neural pathways involved in motor control, such as the corticospinal and reticulospinal tracts.
As a research intern under Dr. Jennifer Gutterman, I created a categorized database that examined 57 studies linking DW-MRI imaging methods to motor outcomes in stroke patients. This work involved analyzing how changes in specific brain pathways relate to patients' motor abilities over time. The findings will support future research and contribute to a literature review by Dr. Gutterman. Currently, I am collaborating with Mr. Michael Glassen to analyze clinical data from the lab, focusing on assessing the motor conditions of patients.