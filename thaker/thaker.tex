\documentclass[12pt,conference,onecolumn]{IEEEtran}

\title{Computational solver for studying kingfisher dive dynamics }
\author{Pooja Thaker}
\date{January 28. 2025}

\newcommand{\keywords}{kingfisher, Alcedinidae, biomechanics, computational fluid dynamics, plunge diving, ANSYS, immersed boundary, IBAMR, IB2d, volume-of-fluid, VOF, overset grid}

\usepackage{hyperref}
\makeatletter
\AtBeginDocument{
\hypersetup{%
pdftitle={\@title},
pdfauthor={\@author},
pdfkeywords={\keywords}}}
\makeatother

\begin{document}
\maketitle 

\begin{abstract}
My research investigates the fluid-structure interactions during kingfisher dives using computational simulations to understand better how they avoid injury. Kingfishers (Alcedinidae) are plunge-diving birds that dive at high speeds, involving a significant impact against the water surface with forces sufficient to cause concussions in humans. Previous research in mechanical exploration is seen with gannets (sulidae) on their neck dynamics and previous research kingfishers focuses on genetic factors. I am focusing on the immersed boundary method, which combines and overlays a Lagrangian (Solid) frame over an Eulerian (Fluid) frame to simulate fluid-structure interactions. Previous research contains little work on the immersed boundary when studying contact-line dynamics, the physics when a solid meets a liquid, at the multiphase level, and when simulations/models involve more than 1 fluid, on complex geometries. 

I am investigating the intersection of the immersed boundary method and Eulerian multiphase modeling to understand life in motion at fluid entry. I am using ANSYS Fluent, which supports the Volume of Fluid (VOF) model, a multiphase model that tracks the interface between fluids. ANSYS Fluent supports Overset Meshing instead of immersed boundary. Overset increases the precision of calculations and is especially helpful for dynamic multi-phased fluids compared to immersed boundaries. Current efforts focus on integrating the immersed boundary method with an overset framework, potentially using a non-conforming overset boundary within a Cartesian grid. My short-term goals include running programmed simulations, and my long-term goals include data visualization and physics analysis to compare drag forces during the dive to that of a toy object. The study
\end{abstract}

\begin{IEEEkeywords}
\keywords
\end{IEEEkeywords}

\end{document}
