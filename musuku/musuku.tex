\documentclass[12pt,conference,onecolumn]{IEEEtran}

\title{RescueVision: augmented reality solutions for first responders}
\author{
\IEEEauthorblockN{Shreyas Musuku}\IEEEauthorblockA{Science \& Engineering\\Manalapan High School\\Englishtown, NJ} \and 
\IEEEauthorblockN{Jason Wong}\IEEEauthorblockA{High Technology High School\\Lincroft, NJ} \and 
\IEEEauthorblockN{Bhargav Yeramsetty}\IEEEauthorblockA{High Technology High School\\Lincroft, NJ}
}
\date{January 28. 2025}

\newcommand{\keywords}{augmented reality, heads up display, HUD, first responder, internship, AT\&T, FirstNet, cloud-based processing}

\usepackage{hyperref}
\makeatletter
\AtBeginDocument{
\hypersetup{%
pdftitle={\@title},
pdfauthor={Shreyas Musuku, Jason Wong, and Bargav Yeramsetty},
pdfkeywords={\keywords}}}
\makeatother

\begin{document}
\maketitle 

\begin{abstract}
Natural disasters and emergencies have caused significant casualties annually, posing severe risks to both civilians and first responders. Despite advancements in technology that could help do so, current tools fail to provide the real-time situational awareness and safety enhancements needed in life-threatening scenarios. RescueVision addresses this gap by integrating augmented reality (AR), thermal imaging, and real-time hazard detection into a lightweight, hands-free heads-up display (HUD) for first responders.

The system enhances visibility in smoke-filled or low-visibility environments using infrared and thermal imaging while providing instant environmental data and structural risk analysis. Powered by AT\&T FirstNet and cloud-based processing, RescueVision ensures seamless communication and decision-making during high-pressure situations. By equipping firefighters, paramedics, and disaster relief organizations with these tools, RescueVision aims to save lives, improve safety, and optimize firefighter efficiency in critical moments.
\end{abstract}

\begin{IEEEkeywords}
\keywords
\end{IEEEkeywords}
\end{document}