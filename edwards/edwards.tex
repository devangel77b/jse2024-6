\documentclass[12pt,conference,onecolumn]{IEEEtran}

\title{Multiple neural network system for early congestive heart failure detection}
\author{\IEEEauthorblockN{Ryan Edwards}\IEEEauthorblockA{Science \& Engineering\\Manalapan High School\\Englishtown, NJ\\425redwards@frhsd.com}}
\date{January 28, 2025}

\newcommand{\keywords}{neural network, congestive heart failure, ANN, artificial neural network, CNN, convolutional neural network, magentic resonance imaging, risk factor analysis, diagnosis, diagnostic AI}

\usepackage{hyperref}
\makeatletter
\AtBeginDocument{
\hypersetup{%
pdftitle={\@title},
pdfauthor={Ryan Edwards},
pdfkeywords={\keywords}}}
\makeatother

\begin{document}
\maketitle 

\begin{abstract}
A disease currently found in roughly 6.5 million people under 20 years old, Congestive Heart Failure (CHF) is cardiovascular disease that causes issues with the blood flow of the left ventricle of the heart. CHF is often mistaken for common aches and pains, and due to the inconvenience of testing for it, CHF often goes undetected until the patient is far past any form of treatment. As there is no current cure for CHF, it is best for doctors to find it early, so they can dampen its effects. To provide a more convenient method of diagnosing CHF, while maintaining the accuracy of medical professionals, a stacking neural network system was employed. This system utilized an Artificial Neural Network (ANN) in order to complete a risk factor analysis (where the risk factors were age, anemia, high blood pressure, etc.) as well as a Convolutional Neural Network (CNN) to complete an image analysis of the patient’s Magnetic Resonance Imaging (MRI) data. These two neural networks would work in tandem with each other to form an accurate diagnosis of CHF in any given patient.
\end{abstract}

\begin{IEEEkeywords}
\keywords
\end{IEEEkeywords}

\end{document}