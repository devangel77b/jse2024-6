\documentclass[12pt,conference,onecolumn]{IEEEtran}

\title{Matrix New World Engineering Internship}
\author{Ryan Cohen}
\date{January 28, 2025}

\begin{document}
\maketitle 

\begin{abstract}
During my internship at Matrix New World Engineering, an engineering firm based in Eatontown, New Jersey, I was able to gain insight in several civil and environmental engineering projects. The firm focuses on environmental services, urban revitalization, and suitable infrastructure, working on a variety of projects across public and private sectors. For the past semester, I spent most of my time focusing on two major projects. The first was the NYC Pier 6 Rehabilitation Program, a reconstruction project located in Brooklyn, New York. I was given the task of finding areas, cross-checking dimensions, and filling out quantities. I also contributed to the actual design of the pier, creating timber caps for retaining walls and other structures. Several of these tasks featured specific functions in CAD such as the hatch feature, which over the course of my internship I learned how to use. The second project was the Garfield New Elementary School, located in Garfield, New Jersey. I went through the process of checking callouts, ensuring they were accurate and consistent with project specifications from the contractors. I verified addresses, dimensions, and other details in order to maintain organization across the details given from contractors and our own details. My internship not only allowed me to improve upon my skills with CAD but also allowed me to develop a more thorough understanding of the step by step process behind large-scale engineering projects. The experience I had there gave me a clearer understanding of the several types of projects that many civil and structural engineers are tasked with on a daily basis.
\end{abstract}

\begin{IEEEkeywords}
Matrix New World Engineering, internship, civil engineer, architecture, professional engineer, engineer-in-training, NYC Pier 6 Rehabilitation, Garfield New Elementary School, CAD, structural engineering
\end{IEEEkeywords}

\end{document}