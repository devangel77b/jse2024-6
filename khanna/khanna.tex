\documentclass[12pt,conference,onecolumn]{IEEEtran}

\title{Ultrasonic device for rapid air bubble removal in automotive coolant systems}
\author{%
\IEEEauthorblockN{Anirudh Khanna}\IEEEauthorblockA{Science \& Engineering\\Manalapan High School\\Englishtown, NJ\\425akhanna@frhsd.com} \and 
\IEEEauthorblockN{Nareshsanjay Muthukumar}\IEEEauthorblockA{Science \& Engineering\\Manalapan High School\\Englishtown, NJ\\425nmuthukumar@frhsd.com}}
\date{January 28, 2025}

\newcommand{\keywords}{ultrasound, bubbles, automotive, coolant system, degas, feed-and-bleed, fluid system, noncondensable gases, air-intrusion, piping system}

\usepackage{siunitx}
\usepackage{hyperref}
\makeatletter
\AtBeginDocument{
\hypersetup{%
pdftitle={\@title},
pdfauthor={Anirudh Khanna and Nareshsanjay Muthukumar},
pdfkeywords={\keywords}}}
\makeatother

\begin{document}
\maketitle 

\begin{abstract}
The efficiency and longevity of automotive engines are heavily influenced by the proper operation of cooling systems, which can be compromised by the presence of air bubbles. Traditional methods of removing air bubbles, such as system bleeding, are time-consuming and often ineffective. This project introduces an ultrasonic device designed to rapidly and efficiently eliminate air bubbles from automotive coolant systems. Using an LM555 timer chip and push-pull output stage to develop \qty{40}{\kilo\hertz} ultrasound, the device powers a \qty{60}{\watt} ultrasonic transducer that emits sound waves to promote the collapse, coalescence, and release of air bubbles and entrained gasses within the coolant system. Preliminary tests show that this ultrasonic method is faster and more efficient than conventional techniques, while ensuring compatibility with existing automotive systems. The device potentially improves engine efficiency by enhancing coolant flow and reducing the risk of overheating due to trapped air. Potential benefits include faster maintenance, reduced labor costs, and enhanced system performance.
\end{abstract}

\begin{IEEEkeywords}
\keywords
\end{IEEEkeywords}

\end{document}